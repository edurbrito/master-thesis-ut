% Institute of Computer Science thesis template
% authors: Sven Laur, Liina Kamm, Tõnu Tamme
% last change Eero Vainikko <eero.vainikko@ut.ee> 12.01.2021
%--
% Compilation instructions:
% 1. Choose main language on line 55-56 (English or Estonian)
% 2. Compile 1-3 times to get refences right
% pdflatex unitartucs-thesis-template
% bibtex unitartucs-thesis-template
%--
% Please use references like this:
% <text> <non-breaking-space> <cite/ref-command> <punctuation>
% This is an example~\cite{example}.

\documentclass[12pt]{article}

% A package for setting layout and margins for your thesis 
\usepackage[a4paper]{geometry}

%%=== A4 page setup ===
%\setlength{\paperwidth}{21.0cm} 
%\setlength{\paperheight}{29.7cm}
%\setlength{\textwidth}{16cm}
%\setlength{\textheight}{25cm}


% When you write in Estonian then you want to use text with right character set
% By default LaTeX does not know what to do with õäöu letters. You have to specify
% a correct input and font encoding. For that you have to Google the Web     
%
% For TexShop under MacOS X. The right lines are 
%\usepackage[applemac]{inputenc}
%\usepackage[T1]{fontenc} %Absolutely critical for *hyphenation* of words with non-ASCII letters.
%
% For Windows and Linux the right magic lines are   
% \usepackage[latin1]{inputenc}
% \usepackage[latin5]{inputenc}
%
\usepackage[utf8]{inputenc} %standard encoding since 2018 (can be commented out?)
\usepackage[T1]{fontenc} %Absolutely critical for *hyphenation* of words with non-ASCII letters.

% Typeset text in Times Roman instead of Computer Modern (EC)
\usepackage{times}

% Suggested packages:
\usepackage{microtype}  %towards typographic perfection...
\usepackage{inconsolata} %nicer font for code listings. (Use \ttfamily for lstinline bastype)


% Use package babel for English or Estonian 
% If you use Estonian make sure that Estonian hyphenation is installed 
% - hypen-estonian or eehyp packages
%
%===Choose the main language in thesis
\usepackage[estonian, english]{babel} %the thesis is in English 
%\usepackage[english, estonian]{babel} %the thesis is in Estonian

% Change Babel document elements 
\addto\captionsestonian{%
  \renewcommand{\refname}{Viidatud kirjandus}%
  \renewcommand{\appendixname}{Lisad}%
}

% If you have problems with Estonian keywords in the bibliography
%\usepackage{biblatex}
%\usepackage[backend=biber]{biblatex}
%\usepackage[style=alphabetic]{biblatex}
%% plain --> \usepackage[style=numeric]{biblatex}
%% abbrv --> \usepackage[style=numeric,firstinits=true]{biblatex}
%% unsrt --> \usepackage[style=numeric,sorting=none]{biblatex}
%% alpha --> \usepackage[style=alphabetic]{biblatex}
%\DefineBibliographyStrings{estonian}{and={ja}}
%\addbibresource{unitartucs-thesis.bib}


% General packages for math in general, theorems and symbols 
% Read ftp://ftp.ams.org/ams/doc/amsmath/short-math-guide.pdf for further information
\usepackage{amsmath} 
\usepackage{amsthm}
\usepackage{amssymb}

% Optional calligraphic fonts    
% \usepackage[mathscr]{eucal}

% Print a dot instead of colon in table or figure captions
\usepackage[labelsep=period]{caption}

% Packages for building tables and tabulars 
\usepackage{array}
\usepackage{tabu}   % Wide lines in tables
\usepackage{xspace} % Non-eatable spaces in macros

% Including graphical images and setting the figure directory
\usepackage{graphicx}
\graphicspath{{figures/}}

% Packages for getting clickable links in PDF file
%\usepackage{hyperref}
\usepackage[hidelinks]{hyperref} %hide red (blue,green) boxes around links
\usepackage[all]{hypcap}


% Packages for defining colourful text together with some colours
\usepackage{color}
\usepackage{xcolor} 
%\definecolor{dkgreen}{rgb}{0,0.6,0}
%\definecolor{gray}{rgb}{0.5,0.5,0.5}
\definecolor{mauve}{rgb}{0.58,0,0.82}


% Standard package for drawing algorithms
% Since the thesis in article format we must define \chapter for
% the package algorithm2e (otherwise obscure errors occur) 
\let\chapter\section
\usepackage[ruled, vlined, linesnumbered]{algorithm2e}

% Fix a  set of keywords which you use inside algorithms
\SetKw{True}{true}
\SetKw{False}{false}
\SetKwData{typeInt}{Int}
\SetKwData{typeRat}{Rat}
\SetKwData{Defined}{Defined}
\SetKwFunction{parseStatement}{parseStatement}


% Nice todo notes
\usepackage{todonotes}

% comments and verbatim text (code)
\usepackage{verbatim}


% Proper way to create coloured code listings
\usepackage{listings}
\lstset{ 
  %language=python,                % the language of the code
  language=C++,
  basicstyle=\footnotesize,        % the size of the fonts that are used for the code
  %numbers=left,                   % where to put the line-numbers
  %numberstyle=\footnotesize,      % the size of the fonts that are used for the line-numbers
  numberstyle=\tiny\color{gray}, 
  stepnumber=1,                    % the step between two line-numbers. If it's 1, each line 
                                   % will be numbered
  numbersep=5pt,                   % how far the line-numbers are from the code
  backgroundcolor=\color{white},   % choose the background color. You must add \usepackage{color}
  showspaces=false,                % show spaces adding particular underscores
  showstringspaces=false,          % underline spaces within strings
  showtabs=false,                  % show tabs within strings adding particular underscores
  frame = lines,
  %frame=single,                   % adds a frame around the code
  rulecolor=\color{black},		   % if not set, the frame-color may be changed on line-breaks within 
                                   % not-black text (e.g. commens (green here))
  tabsize=2,                       % sets default tabsize to 2 spaces
  captionpos=b,                    % sets the caption-position to bottom
  breaklines=true,                 % sets automatic line breaking
  breakatwhitespace=false,         % sets if automatic breaks should only happen at whitespace
  %title=\lstname,                 % show the filename of files included with \lstinputlisting;
                                   % also try caption instead of title
  keywordstyle=\color{blue},       % keywCurriculumord style
  commentstyle=\color{dkgreen},    % comment style
  stringstyle=\color{mauve},       % string literal style
  escapeinside={\%*}{*)},          % if you want to add a comment within your code
  morekeywords={*,game, fun}       % if you want to add more keywords to the set
}


% Obscure packages to write logic formulae and program semantics
% Unless you do a thesis on program semantics or static code analysis you do not need that
% http://logicmatters.net/resources/ndexamples/proofsty3.html <= writing type rules => use semantic::inference
% ftp://tug.ctan.org/tex-archive/macros/latex/contrib/semantic/semantic.pdf
% \usepackage{proof}
% \usepackage{semantic} 
% \setlength{\inferLineSkip}{4pt}
% \def\predicatebegin #1\predicateend{$\Gamma \vdash #1$}

% If you really want to draw figures in LaTeX use packages tikz or pstricks
% However, getting a corresponding illustrations is really painful  


% Define your favorite macros that you use inside the thesis 
% Name followed by non-removable space
\newcommand{\proveit}{ProveIt\xspace}

% Macros that make sure that the math mode is set
\newcommand{\typeF}[1] {\ensuremath{\mathsf{type_{#1}}}\xspace}
\newcommand{\opDiv}{\ensuremath{\backslash \mathsf{div}}\xspace} 

% Nice Todo box
\setlength{\marginparwidth}{2cm}
\newcommand{\TODO}{\todo[inline]}

% A way to define theorems and lemmata
\newtheorem{theorem}{Theorem}



%%% BEGIN DOCUMENT
\begin{document}

%===BEGIN TITLE PAGE
\thispagestyle{empty}
\begin{center}

\large
\iflanguage{english}{%
UNIVERSITY OF TARTU\\
Faculty of Science and Technology\\
Institute of Computer Science\\
Computer Science Curriculum\\
%Software Engineering Curriculum\\
}{%\iflanguage
TARTU ÜLIKOOL\\
Loodus- ja täppisteaduste valdkond\\
Arvutiteaduse instituut\\
Informaatika õppekava\\
}%\iflanguage

%\vspace*{\stretch{5}}
\vspace{25mm}

\Large Alice Cooper

\vspace{4mm}

\huge Type Inference for Fourth Order Logic Formulae 

%\vspace*{\stretch{7}}
\vspace{20mm}

\Large
\iflanguage{english}{%
%Bachelor's Thesis (9 ECTS)
Master's Thesis (30 ECTS)
}{%\iflanguage
Bakalaureusetöö (9 EAP)
}%\iflanguage

\end{center}

\vspace{2mm}

\begin{flushright}
 {
 \setlength{\extrarowheight}{5pt}
 \begin{tabular}{r l} 
  \sffamily \iflanguage{english}{Supervisor(s)}{Juhendaja(d)}: &                    \sffamily Axel Rose, MSc \\
             & \sffamily May Flower, PhD
 \end{tabular}
 }
\end{flushright}

%\vspace*{\stretch{3}}\iflanguage
%\vspace{10mm}

\vfill
\centerline{\large Tartu \the\year}

%===END TITLE PAGE

% If the thesis is printed on both sides of the page then 
% the second page must be must be empty. Comment this out
% if you print only to one side of the page comment this out
%\newpage
%\thispagestyle{empty}    
%\phantom{Text to fill the page}
% END OF EXTRA PAGE WITHOUT NUMBER


%===COMPULSORY INFO PAGE
\newpage

%=== Info in English
\newcommand\EngInfo{{%
\selectlanguage{english}
\noindent\textbf{\large Type Inference for Fourth Order Logic Formulae}

\vspace*{3ex}

\noindent\textbf{Abstract:}

\noindent
Many interpreting program languages are dynamically typed, such as Visual Basic or Python. As a result, it is easy to write programs that crash due to mismatches of provided and expected data types.  One possible solution to this problem is automatic type derivation during compilation. In this work, we consider study how to detect type errors in the \textsc{Whitespace} language by using fourth order logic formulae as annotations. The main result of this thesis is a new triple-exponential type inference algorithm for the fourth order logic formulae. This is a significant advancement as the question whether there exists such an algorithm was an open question. 
All previous attempts to solve the problem lead lead to logical inconsistencies or required tedious user interaction in terms of interpretative dance. Although the resulting algorithm is slightly inefficient, it can be used to detect obscure programming bugs in the \textsc{Whitespace} language. The latter significantly improves productivity. Our practical experiments showed that productivity is comparable to average Java programmer.   
From a theoretical viewpoint, the result is only a small advancement in rigorous treatment of higher order logic formulae. The results obtained by us do not generalise to formulae with the fifth or higher order. 

\vspace*{1ex}

\noindent\textbf{Keywords:}\\
\TODO{List of keywords}
%Layout, formatting, template

\vspace*{1ex}

\noindent\textbf{CERCS:}\TODO{CERCS code and name:~\url{https://www.etis.ee/Portal/Classifiers/Details/d3717f7b-bec8-4cd9-8ea4-c89cd56ca46e}}

\vspace*{1ex}
}}%\newcommand\EngInfo


%=== Info in Estonian
\newcommand\EstInfo{{%
\selectlanguage{estonian}
\noindent\textbf{\large Tüübituletus neljandat järku loogikavalemitele}
\vspace*{1ex}

\noindent\textbf{Lühikokkuvõte:} 

%\noindent ...

\TODO{One or two sentences providing a basic introduction to the field, comprehensible to a scientist in
any discipline.}
\TODO{Two to three sentences of
more detailed background, comprehensible to scientists in related disciplines.}
\TODO{One sentence clearly stating the general problem being addressed by this particular
study.}
\TODO{One sentence summarising the main result (with the words ``here we show´´ or their equivalent).}
\TODO{Two or three sentences explaining what
the main result reveals in direct
comparison to what was thought to be the case previously, or how the main result adds to previous knowledge.}
\TODO{One or two sentences to put the results into a more general context.}
\TODO{Two or three sentences to provide a
broader perspective, readily
comprehensible to a scientist in any
discipline, may be included in the first paragraph
if the editor considers that the accessibility of
the paper is significantly enhanced by their inclusion.}

\vspace*{1ex}

\noindent\textbf{Võtmesõnad:}\\
\TODO{List of keywords}
%Layout, formatting, template

\vspace*{1ex}

\noindent\textbf{CERCS:}\TODO{CERCS kood ja nimetus:~\url{https://www.etis.ee/Portal/Classifiers/Details/d3717f7b-bec8-4cd9-8ea4-c89cd56ca46e}}

\vspace*{1ex}
}}%\newcommand\EstInfo


%=== Determine the order of languages on Info page
\iflanguage{english}{\EngInfo}{\EstInfo}
\iflanguage{estonian}{\EngInfo}{\EstInfo}


\newpage
\tableofcontents


% Remember to remove this from the final thesis version
\newpage
\listoftodos[Unsolved issues]
% END OF TODO PAGE 


\newpage
\section{Introduction}

% \TODO{What is it in simple terms (title)?}
% \TODO{Why should anyone care?}
% \TODO{What was my contribution?} 
% \TODO{What you are doing in each section (a sentence or two per section)}
One is where and when one claims to be - this is the underlying principle of most of today's location-based services. This principle, however, hides a whole set of preceding premises that implicitly assert trust in the subject's honesty in reporting its correct location. Having this trust delegated, the reliance on a trusted third party, frequently an atomic computing entity, is still subject to tampering, repudiation, inaccuracy, punctual and single failure, or any other kind of Byzantine behaviour. 

The trust levels required to testify to one's alibi remain unmeasurable in modern arrangements. Strategic interactions between rational agents often support this trust. One party provides a location-based service, and another party makes use of it for its individual benefit and by providing a non-tampered time-conscious piece of location claim. This interaction appears to be one of a non-zero-sum game that can be observed in most GPS-based services, mapping platforms, navigation systems, mobility and ride-hailing apps, among many others. If driven by the reasoning goal of extracting correct information from the interacting system, users are logically motivated to report an accurate location. The services, having the higher goal of not losing users due to their reported malfunctioning or inaccuracy, are thus motivated to provide maximized quality when operating and consuming the location claims. 

This paradigm is now ubiquitous, which may lead to its fallacious use in other very distinct scenarios. Those scenarios are, therefore and inversely, the ones that fundamentally require verifiable proof of location to assert a particular state or derive a conclusion. Consider, for example, scenarios requiring location-based authentication or authorization in adversarial environments that rely on information gathered in a trustless setup. These materialize into services requiring, for instance, a digital certificate as proof that a given user is within a particular geographical area, to enable certain functionalities or assert liability, as in location-based access control, review or reward systems, augmented reality games, social networks, etc... Security against geo-tampering or location spoofing in a relatively trustless environment is needed to achieve the required integrity. 

The basic infrastructural concept is somewhat understood, and theoretical or experimental solutions have been delivered throughout the years. These solutions have evolved parallel with their trust assumptions, beginning with a fully trusted setup and progressively shifting towards modern requirements for operational decentralization - of power and profit. Most recent attempts contemplate the need for a permissionless means of reaching consensus between a quorum of witnesses that can attest to one's presence at a given point in space and at a given moment in time. These concepts take shape with a combination of tools: wireless technologies as message-exchanging means, cryptographic protocols as confidentiality, integrity, or authentication enablers, and distributed ledgers as publicly trusted record keepers. 

The quest for a solution that could make these location-based services as prevalent and ubiquitous shall aim to address a set of design challenges. These challenges are, among others, the solution's flexibility and deployability, preferably by making use of existing infrastructure, or at least accessible technology, and the solution's security and privacy, obeying the modern cryptographic standards and requirements, to guarantee some level of privacy, and resiliency to attacks. This thesis, aiming to address these matters, delivers the following contributions:
\begin{enumerate}
\item A semi-formalization of the location-based services paradigm, including the underlying premises and the strategic interactions between rational agents, along with a review of the state of the art in the field. The review is discriminated in terms of trust levels, from fully trusted to permissionless environments, and in terms of the underlying technology, from centralized to decentralized.
\item The design and implementation of a proof-of-concept that can be deployed in a permissionless manner, using existing infrastructure, and that can be used to attest to one's presence at a given point in space and time. Specifically, the proof-of-concept is based on the use of routing protocols for multi-hop mobile ad hoc networks to set up a mesh network of witnesses that can attest to one's presence in a given geographical area.
\end{enumerate}

The structure of the work is as follows. In chapter 2, an introduction to the underlying concepts and hypotheses is provided, as well as a mention to the technology involded in the practical implementation. Chapter 3 examines and analyzes similar work discriminated in terms of trust levels. In chapter 4, a general overview of the requirements for the proposed solution is given. Chapter 5 details the architecture's design, implementation, and evaluation. Finally, chapter 6 presents the conclusion and recommendations for future work.

\newpage
\section{Prelude} 

This section introduces not only the underlying concepts that sustent the work, but also the technology involved in implementing the proposed proof of concept.

\subsection{Proof-of-Location}



\subsubsection{Parties Involved}

The general act of witnessing alludes to the simultaneous spatiotemporal existence of a set of entities with distinct roles. The majority of the protocols convey a clear distinction between these roles, highlighting the relative dynamism that distinguishes those entities. 

In comparable terms, highly dynamic entities do not maintain a fixed geographical location for long periods of time. They are often observed in movement, thereby repeatedly starting and finishing communication procedures with nearby entities. On the other hand, static entities are expected not to engage in frequent position changes, expressing continuous and fairly invariable communication availability around a fixed point in space as time passes \cite{nasrulin2018robust}. The act is, however, only completed with another type of entity from whom neither the relative staticity nor the relative dynamism frankly matters. These protocol parties are often external and asynchronous to the witnessing process, but they do effectively take a non-negligible part in incentivizing and giving significance to the witnessing act. 

Concisely and in concrete terms, these location-proof arrangements expect the existence of a \emph{prover} that engages in any communication protocol with nearby participants, the \emph{witnesses}, with the goal of gathering a verifiable proof-of-location claim, to be later presented to a \emph{verifier}, therefore convincing it of one's existence within a geographical area at a given moment in the past \cite{dupin2018location}.

\paragraph{Prover.} A prover is a dynamic entity, both in movement and availability terms, that is expected to be able to communicate with the witnesses to gather a proof of its location and to be later able to provide a location claim to the verifier. Communication with nearby witnesses is thought to happen wirelessly, using any short-range message transmission means. Provers are also expected to be associated with a verifiable but desirably private identity, often as a pseudonym.

\paragraph{Witness.} A witness is adjunctly an entity that is expected to be able to communicate with the prover via the same short-range communication channel and to be able to provide it with a verifiable piece of location attestation. The witnesses are envisioned to seldomly change their absolute location and maintain a relatively stable neighbouring list of nearby witnesses. These references aim at attaining the figurative creation of coverage zones as strongly connected graphs that form the boundaries of the atomic units of a polygonal mesh. Witnesses are as well expected to be identified, usually by a pseudonym.

\paragraph{Verifier.} A verifier is an external entity that is expected to be able to receive a location claim from a prover and to be able to verify its validity. Even though possible and predicted for trusted setups, in a trustless environment and with the general assurances of a permissionless protocol, verifiers shall not have the need to communicate directly with the witnesses. Verifiers' identity is also of no measurable importance for the protocol, as the interaction between the prover and the verifier is usually asynchronous and external to the witnessing process.

\subsubsection{Adversary Models}

\subsection{Wireless Mesh Networks}

The envisioned fourth industrial revolution has set the track for modern advancements in achieving a global web of pervasive connectivity between all sorts of machines \cite{akyildiz2005wireless, cilfone2019wireless}. New means of radio and wireless communication have been pushing for the technological heterogeneity of protocols, architectures, devices, and consequent performance levels in order to find their design suitability for different coverage or range scenarios, transmission or bandwidth rates \cite{sichitiu2005wireless}. Additionally, requirements for more complex, adaptable, and resilient topologies have captured broad academic and industry interest \cite{cilfone2019wireless}. 

\begin{figure} [ht]
  \begin{center}
  \includegraphics[width=0.9\textwidth]{mesh-network-topology}
  \caption{Different topologies of a computer network.}
  \label{fig:mesh-network-topology}
  \end{center}
\end{figure}

\TODO{Draw my own image for network topologies.}

The development of new hardware architectures, protocols and applications started gaining momentum and branched their way forward to support the rising popularisation of Wireless Mesh Networks (WMNs). In mesh topologies (see Fig. \ref{fig:mesh-network-topology}), network nodes are directly and dynamically connected in a non-hierarchical way. This trait eventually allows for many-to-many communications between the devices to efficiently route data from a generic source to a generic destination. The infrastructure nodes that make up the mesh are expected to dynamically self-organize and configure themselves, resulting in beneficial distributed effects on the overall fault tolerance, ease of deployment, and workload allocation \cite{cilfone2019wireless, sichitiu2005wireless}. WMNs follow these principles with the particularity of being made up of radio nodes that communicate via any sort of wireless technology.

Some of the most common wireless technologies that have been, throughout the years, ported to WMNs are IEEE 802.11, Bluetooth, IEEE 802.15, and LoRa. The first is the most popular and widely used, being the basis for the Wi-Fi standard, which, at the beginning of the last decade, saw an amendment that mainly targeted mesh networks, the IEEE 802.11s WLAN Mesh Standard \cite{hiertz2010ieee}. The novelty came with the introduction of routing mechanisms operating at the ISO/OSI Layer 2, allowing for compatible information delivery in the layers above. The dynamic establishment of a topology for IEEE 802.11s-based mesh networks relies on the phased transmission of beacon messages that allow for the discovery, synchronization, and maintenance of the links between the peers. IEEE 802.11s has a default routing protocol, the Hybrid Wireless Mesh Protocol (HWMP), which is based on a series of flooding procedures for both proactive and reactive path finding and selection \cite{bari2012performance}. This protocol is, however, not strictly enforced by the standard and has been replaced by other, more popular solutions. One notable example is the Better Approach To Mobile Ad-hoc Networks (B.A.T.M.A.N.) routing protocol.

This thesis will explore the concept of WMNs and their potential for serving as the infrastructural topology that enables the relatively short-ranged wireless exchange of messages between the participants of a proof-of-location protocol. The following sections will present the B.A.T.M.A.N. routing protocol, OpenWRT and other relevant tools that will be later used to implement the proof of concept.

\subsubsection{B.A.T.M.A.N. Routing Protocol}

The Better Approach To Mobile Ad-hoc Networks (B.A.T.M.A.N.)\footnote{http://www.open-mesh.org/} is a proactive routing protocol for WMNs that operates not at the network layer but at the transport layer, by asserting the reliability of radio links using routing metrics and a distance-vector approach \cite{seither2011routing}. Its newer wireless version, \emph{batman-adv}, has gained traction and popularity and eventually made itself available in the Linux kernel.

Route discovery is preemptively replaced with neighbour discovery, and each infrastructural node is instructed to calculate its potential best next-hop, significantly reducing the overhead of requiring each peer to be aware of the whole network topology. Its version V introduced a throughput metric to evaluate the links' quality and routing choices, replacing the version IV packet loss-based metric, deemed unsuitable for larger network sizes \cite{seither2011routing}.

\begin{figure} [ht]
  \begin{center}
  \includegraphics[width=0.9\textwidth]{batman-adv-ogmv2}
  \caption{OriGinator Message version 2 (OGMv2) packet format \cite{cilfone2019wireless,open-mesh-ogmv2}.}
  \label{fig:batman-adv-ogmv2}
  \end{center}
\end{figure}

The discovery of neighbouring nodes is accomplished by broadcasting OriGinator Messages (OGMv2, see Fig. \ref{fig:batman-adv-ogmv2}), featuring a collision avoidance delay mechanism, detection of new and duplicate messages, and other fields for throughput measurement and gateway discovery \cite{cilfone2019wireless}. Hence, the OGM flooding protocol enables mesh routing procedures that, simultaneously but independently, allow for estimating the quality of the individual links. Additionally, the protocol enables OGM aggregations as an effort to reduce the overhead of sending many short-sized frames. Nevertheless, there is still a quest for optimizations that would allow for more efficient use of multiple interfaces. An implementation of a subset of the Internet Control Message Protocol (ICMP) was also made available, allowing, for instance, the use of the \emph{ping} command to test the connectivity between nodes \cite{seither2011routing}.

\subsubsection{OpenWRT, QEMU, and Raspberry Pis}

\subsection{Permissionless Consensus}

\subsubsection{Proof-of-Work and Proof-of-Stake}

\subsubsection{Proof-of-X}

\subsection{Title of Subsection 2} 

Rule: If you divide the text into subsections (or subsubsections) then there has to be at least two of them, otherwise do not create any. 

Tip: You can also use paragraphs, e.g.
\paragraph{Type rules for integers.} Some text ...

\paragraph{Type rules for rational numbers.} Some text here too...




\subsection{How to use references} \label{sec:using_ref}

\paragraph{Cross-references to figures, tables and other document elements.}
LaTeX  internally numbers all kind of objects that have sequence numbers:
\begin{itemize}
\item chapters, sections, subsections;
\item figures, tables, algorithms;
\item equations, equation arrays.
\end{itemize}
To reference them automatically, you have to generate a label using \texttt{$\backslash$label\{some-name\}} just after the object that has the number inside. Usually, labels of different objects are split into different namespaces by adding dedicated prefix, such as \texttt{sec:}, \texttt{fig:}. To use the corresponding reference, you must use command \texttt{$\backslash$ref} or \texttt{$\backslash$eqref}. For instance, we can reference this subsection by calling Section~\ref{sec:using_ref}. Note that there should be a nonbreakable space \texttt{\~} between the name of the object and the reference so that they would not appear on different lines (does not work in Estonian).          



\paragraph{Citations.}
Usually, you also want to reference articles, webpages, tools or programs or books. For that you should use citations and references. The system is similar to the cross-referencing system in LaTeX. For each reference you must assign a unique label. Again, there are many naming schemes for labels. However, as you have a short document anything works. To reference to a particular source you must use \texttt{$\backslash$cite\{label\}} or \texttt{$\backslash$cite[page]\{label\}}. 

References themselves can be part of a LaTeX source file. For that you need to define a bibliography section. However, this approach is really uncommon. It is much more easier to use BibTeX to synthesise the right reference form for you. For that you must use two commands in the LaTeX source
\begin{itemize}
\item $\backslash$bibliographystyle\{alpha\} or $\backslash$bibliographystyle\{plain\}
\item $\backslash$bibliography\{file-name\}
\end{itemize}
The first command determines whether the references are numbered by letter-number combinations or by cryptic numbers. It is more common to use \texttt{alpha} style. The second command determines the file containing the bibliographic entries. The file should end with \texttt{bib} extension. Each reference there is in specific form. The simplest way to avoid all technicalities is to use graphical frontend  Jabref (\url{http://jabref.sourceforge.net/}) to manage references. Another alternative is to use DBLP database of references and copy BibTeX entries directly form there.   
    
   
The following paragraph shows how references can be used. Game-based proving is a way to analyse security of a cryptographic protocol~\cite{GameB_1, GameB_2}. There are automatic provers, such as {CertiCrypt\-}~\cite{certicrypt} and ProVerif~\cite{proVerif}.



\newpage
\section{How to add figures and pictures to your thesis}


Here are a few examples of how to add figures or pictures to your thesis (see Figures~\ref{fig:fnCompModel}, \ref{fig:game-based_proofs}, \ref{fig:proveit_screenshot}).

Rule: All the figures, tables and extras in the thesis have to be referred to somewhere in the text.


\begin{figure} [ht] %try to place the figure here (next option top of the page) 
\begin{center}
\includegraphics[width=0.8\textwidth]{computational_model_function}
\caption{The title of the Figure.}
\label{fig:fnCompModel}
\end{center}
\end{figure}



\begin{figure} [!ht] %if [h] doesn't work, we can force with !
\begin{center}
\includegraphics[width=\textwidth]{game-based_proofs}
\caption{Refer if the figure is not yours~\cite{kamm12}.}
\label{fig:game-based_proofs}
\end{center}
\end{figure}


\begin{figure} [p]
\begin{center}
\includegraphics[width=\textwidth]{proveit_screenshot}
\caption{Screenshot of \proveit.}
\label{fig:proveit_screenshot}
\end{center}
\end{figure}

Tip: If you add a screenshot then labeling the parts might help make the text more understandable (panel C vs bottom left part), e.g.


\begin{figure} [htbp]
\begin{tabular}{c c}
%
\begin{minipage}{0.45\textwidth}
\includegraphics[width=\textwidth]{LCA_2_solutions}
\end{minipage}
%
&
\begin{minipage}{0.55\textwidth}
\centering
\begin{tabular}{ l | l |}
	Node & Decendants \\ \hline
  1 & 2, 3, 4 \\ \hline
  2 & 3, 4 \\ \hline
  3 & \\ \hline
  4 & \\ \hline
  5 & 3, 4, 6, 7 \\ \hline
  6 & 4 \\ \hline
  7 & 3 \\  \hline
  8 & 3, 4, 5, 6, 7\\ \hline
  9 & 3, 4, 5, 6, 7\\ \hline
\end{tabular}
\end{minipage}
\end{tabular}
%
\caption{Example how to put two figures parallel to each other.}
\label{fig:LCA_2_solutions}
\end{figure}


Example: A screenshot of \proveit can be seen on Figure~\ref{fig:proveit_screenshot}. The user first enters the pseudocode of the initial game in panel B. \proveit also keeps track of all the previous games showing the progress on a graph seen in panel A.

There are two figures side by side on Figure~\ref{fig:LCA_2_solutions}.



\clearpage %if newpage doesn't work
\section{Other Ways to Represent Data}

\subsection{Tables}

\begin{table}[h]
\centering
\caption{Statements in the \proveit language.}
\begin{tabular}{| l | l |}
	\hline
	\bf{Statement} & \bf{Typeset Example} \\
	\hline
	assignment & $a := 5 + b$ \\
	\hline
	uniform choice & $m <- M$ \\
	\hline
	function signature & $f : K \times M -> L$\\
	\hline
\end{tabular}
\label{tab:statements}
\end{table}


\subsection{Lists}

Numbered list example:
\begin{enumerate}
	\item item one; 
	\item item two;
	\item item three.
\end{enumerate} 

\subsection{Math mode}
Example:
\begin{equation}
a + b = c + d
\end{equation}
Aligning:
\begin{align*}
	a &= 5 \\
	b + c &= a \\
	a -2*3 &= 5/4
\end{align*}
Hint: Variables or equations in text are separated with \$ sign, e.g. $a$, $x - y$.

% \paragraph{Inference Rules}
% \[ 
% 	\inference[addition]{x : T & y : T}{x + y : T} 
% \]
% Bigger example:
% \[
% \inference[assign]{c := a + b & 
% 	\inference[addG]{a : \typeRat & 
% 		\inference[var]{b : \typeInt & \typeInt \subseteq \typeRat}{b : \typeRat}
% 		}{a + b : \typeRat}
% 	}{c : \typeRat}
% \]


\subsection{algorithm2e}

\begin{algorithm} [!h]
	\caption{typeChecking} \label{alg:typeChecking}
	\KwIn{Abstract syntax tree}
	\KwResult{Type checking result; In addition, type table \typeF{type\_G} for global variables, \typeF{game} for the main game and \typeF{fun} for each $fun \in F$}
	\SetKwData{s}{s}
	\BlankLine
	
	\While{something changed in last cycle}{
		\lForEach{global statement \s} {
			\parseStatement{\s, \typeF{type\_G}}\;
		}
		\ForEach{function $fun$} {
		\lForEach{statement \s in $fun$} {
			\parseStatement{\s, \typeF{fun}}\;
		}
		}
		\lForEach{statement \s in game} {
			\parseStatement{\s, \typeF{game}}\;
		}
	}
	%\eIf{error messages were found}{\Return \False\;}{\Return \True\;}
\end{algorithm}

\subsection{Pseudocode}

\begin{figure} [htb]
\begin{lstlisting}
expression
  : NUMBER
  | VARIABLE
  | '+' expression
  | expression '+' expression
  | expression '*' expression
  | function_name '(' parameters ')'
  | '(' expression ')'
\end{lstlisting}
\caption{Grammar of arithmetic expressions.}
\label{fig:parser_exp}
\end{figure}

\subsection{Frame Around Information}

Tip: We can use minipage to create a frame around some important information.
\begin{figure} [h]
\frame{
\begin{minipage}{\textwidth}
\begin{enumerate}
	\item integer division ($\opDiv$) -- only usable between \typeInt types
	\item remainder ($\%$) -- only usable between \typeInt types
\end{enumerate}
\end{minipage}
}
\caption{Arithmetic operations in \proveit revisited.}
\label{fig:aritmOp_revisit}
\end{figure}



\clearpage
\section{Conclusion}

\TODO{what did you do?} 
\TODO{What are the results?}
\TODO{future work?}

\newpage

% BibTeX bibliography
\bibliographystyle{ieeetr} %plain=[1], alpha=[BGZ09]
\bibliography{unitartucs-thesis}

\addcontentsline{toc}{section}{\refname}


% Use Biblatex if you have problems with Estonian keywords
%\printbibliography %biblatex


% Use alternative local LaTeX bibliography
\begin{comment}
\begin{thebibliography}{9}
\bibitem{proVerif} 
  Bruno Blanchet. 
  Proverif: Cryptographic protocol verifier in the formal model.
  \url{http://www.proverif.ens.fr/}.
  (checked 15.05.2012)
\bibitem{GameB_1} GameB1
\bibitem{GameB_2} GameB2
\bibitem{certicrypt} certicrypt
\bibitem{kamm12} kamm12
\end{thebibliography}
\end{comment}


\newpage
%\appendix
%\section*{\appendixname}
\iflanguage{english}%
  {\section*{Appendix}
  \addcontentsline{toc}{section}{Appendix}
  }%
  {\section*{Lisad}
  \addcontentsline{toc}{section}{Lisad}}


\section*{I. Glossary}
\addcontentsline{toc}{subsection}{I. Glossary}

\newpage

%=== Licence in English
\newcommand{\licencehint}[2]{\\\hspace*{#1}\textsl(#2)\par}
\newcommand\EngLicence{{%
\selectlanguage{english}
\section*{II. Licence}

\addcontentsline{toc}{subsection}{II. Licence}

\subsection*{Non-exclusive licence to reproduce thesis and make thesis public}

I, \textbf{Alice Cooper}, %author's name
  \licencehint{10mm}{author's name}

\begin{enumerate}
\item
herewith grant the University of Tartu a free permit (non-exclusive licence) to
\par
reproduce, for the purpose of preservation, including for adding to the DSpace digital archives until the expiry of the term of copyright,
\par
\textbf{Type Inference for Fourth Order Logic Formulae}, %
  \licencehint{10mm}{title of thesis}
\par
supervised by Axel Rose and May Flower. %supervisor's name
  \licencehint{10mm}{supervisor's name}
\item
I grant the University of Tartu a permit to make the work specified in p. 1 available to the public via the web environment of the University of Tartu, including via the DSpace digital archives, under the Creative Commons licence CC BY NC ND 3.0, which allows, by giving appropriate credit to the author, to reproduce, distribute the work and communicate it to the public, and prohibits the creation of derivative works and any commercial use of the work until the expiry of the term of copyright.
\item
I am aware of the fact that the author retains the rights specified in p. 1 and 2.
\item
I certify that granting the non-exclusive licence does not infringe other persons' intellectual property rights or rights arising from the personal data protection legislation. 
\end{enumerate}

\noindent
Alice Cooper\\ %author's name
\textbf{\textsl{dd/mm/yyyy}}
}}%\newcommand\EngLicence


%=== Licence in Estonian
\newcommand\EstLicence{{%
\selectlanguage{estonian}
\section*{II. Litsents}

\addcontentsline{toc}{subsection}{II. Litsents}

\subsection*{Lihtlitsents lõputöö reprodutseerimiseks ja üldsusele kättesaadavaks tegemiseks}

Mina, \textbf{Alice Cooper}, %author's name
  \licencehint{10mm}{autori nimi}

\begin{enumerate}
\item
annan Tartu Ülikoolile tasuta loa (lihtlitsentsi) minu loodud teose
\par
\textbf{Tüübituletus neljandat järku loogikavalemitele}, %title of thesis
    \licencehint{10mm}{lõputöö pealkiri}
\par
mille juhendaja(d) on Axel Rose ja May Flower, %supervisor's name(s)
  \licencehint{10mm}{juhendaja nimi}
\par
reprodutseerimiseks eesmärgiga seda säilitada, sealhulgas lisada digitaalarhiivi DSpace kuni autoriõiguse kehtivuse lõppemiseni.
\par
\item
Annan Tartu Ülikoolile loa teha punktis 1 nimetatud teos üldsusele kättesaadavaks Tartu Ülikooli veebikeskkonna, sealhulgas digitaalarhiivi DSpace kaudu Creative Commonsi litsentsiga CC BY NC ND 3.0, mis lubab autorile viidates teost reprodutseerida, levitada ja üldsusele suunata ning keelab luua tuletatud teost ja kasutada teost ärieesmärgil, kuni autoriõiguse kehtivuse lõppemiseni.
\item
Olen teadlik, et punktides 1 ja 2 nimetatud õigused jäävad alles ka autorile.
\item
Kinnitan, et lihtlitsentsi andmisega ei riku ma teiste isikute intellektuaalomandi ega isikuandmete kaitse õigusaktidest tulenevaid õigusi. 
\end{enumerate}

\noindent
Alice Cooper\\ %author's name
\textbf{\textsl{pp.kk.aaaa}}
}}%\newcommand\EstLicence


%===Choose the licence in active language
\iflanguage{english}{\EngLicence}{\EstLicence}


\end{document}

