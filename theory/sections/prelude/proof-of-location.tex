
\subsubsection{Parties Involved}

The general act of witnessing alludes to the simultaneous spatiotemporal existence of a set of entities with distinct roles. The majority of the protocols convey a clear distinction between these roles, highlighting the relative dynamism that distinguishes those entities. 

In comparable terms, highly dynamic entities do not maintain a fixed geographical location for long periods of time. They are often observed in movement, thereby repeatedly starting and finishing communication procedures with nearby entities. On the other hand, static entities are expected not to engage in frequent position changes, expressing continuous and fairly invariable communication availability around a fixed point in space as time passes \cite{nasrulin2018robust}. The act is, however, only completed with another type of entity from whom neither the relative staticity nor the relative dynamism frankly matters. These protocol parties are often external and asynchronous to the witnessing process, but they do effectively take a non-negligible part in incentivizing and giving significance to the witnessing act. 

Concisely and in concrete terms, these location-proof arrangements expect the existence of a \emph{prover} that engages in any communication protocol with nearby participants, the \emph{witnesses}, with the goal of gathering a verifiable proof-of-location claim, to be later presented to a \emph{verifier}, therefore convincing it of one's existence within a geographical area at a given moment in the past \cite{dupin2018location}.

\paragraph{Prover.} A prover is a dynamic entity, both in movement and availability terms, that is expected to be able to communicate with the witnesses to gather a proof of its location and to be later able to provide a location claim to the verifier. Communication with nearby witnesses is thought to happen wirelessly, using any short-range message transmission means. Provers are also expected to be associated with a verifiable but desirably private identity, often as a pseudonym.

\paragraph{Witness.} A witness is adjunctly an entity that is expected to be able to communicate with the prover via the same short-range communication channel and to be able to provide it with a verifiable piece of location attestation. The witnesses are envisioned to seldomly change their absolute location and maintain a relatively stable neighbouring list of nearby witnesses. These references aim at attaining the figurative creation of coverage zones as strongly connected graphs that form the boundaries of the atomic units of a polygonal mesh. Witnesses are as well expected to be identified, usually by a pseudonym.

\paragraph{Verifier.} A verifier is an external entity that is expected to be able to receive a location claim from a prover and to be able to verify its validity. Even though possible and predicted for trusted setups, in a trustless environment and with the general assurances of a permissionless protocol, verifiers shall not have the need to communicate directly with the witnesses. Verifiers' identity is also of no measurable importance for the protocol, as the interaction between the prover and the verifier is usually asynchronous and external to the witnessing process.

\subsubsection{Adversary Models}
