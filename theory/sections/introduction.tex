
% \TODO{What is it in simple terms (title)?}
% \TODO{Why should anyone care?}
% \TODO{What was my contribution?} 
% \TODO{What you are doing in each section (a sentence or two per section)}
One is where and when one claims to be - this is the underlying principle of most of today's location-based services. This principle, however, hides a whole set of preceding premises that implicitly assert trust in the subject's honesty in reporting its correct location. Having this trust delegated, the reliance on a trusted third party, frequently an atomic computing entity, is still subject to tampering, repudiation, inaccuracy, punctual and single failure, or any other kind of Byzantine behaviour. 

The trust levels required to testify to one's alibi remain unmeasurable in modern arrangements. Strategic interactions between rational agents often support this trust. One party provides a location-based service, and another party makes use of it for its individual benefit and by providing a non-tampered time-conscious piece of location claim. This interaction appears to be one of a non-zero-sum game that can be observed in most GPS-based services, mapping platforms, navigation systems, mobility and ride-hailing apps, among many others. If driven by the reasoning goal of extracting correct information from the interacting system, users are logically motivated to report an accurate location. The services, having the higher goal of not losing users due to their reported malfunctioning or inaccuracy, are thus motivated to provide maximized quality when operating and consuming the location claims. 

This paradigm is now ubiquitous, which may lead to its fallacious use in other very distinct scenarios. Those scenarios are, therefore and inversely, the ones that fundamentally require verifiable proof of location to assert a particular state or derive a conclusion. Consider, for example, scenarios requiring location-based authentication or authorization in adversarial environments that rely on information gathered in a trustless setup. These materialize into services requiring, for instance, a digital certificate as proof that a given user is within a particular geographical area, to enable certain functionalities or assert liability, as in location-based access control, review or reward systems, augmented reality games, social networks, etc... Security against geo-tampering or location spoofing in a relatively trustless environment is needed to achieve the required integrity. 

The basic infrastructural concept is somewhat understood, and theoretical or experimental solutions have been delivered throughout the years. These solutions have evolved parallel with their trust assumptions, beginning with a fully trusted setup and progressively shifting towards modern requirements for operational decentralization - of power and profit. Most recent attempts contemplate the need for a permissionless means of reaching consensus between a quorum of witnesses that can attest to one's presence at a given point in space and at a given moment in time. These concepts take shape with a combination of tools: wireless technologies as message-exchanging means, cryptographic protocols as confidentiality, integrity, or authentication enablers, and distributed ledgers as publicly trusted record keepers. 

The quest for a solution that could make these location-based services as prevalent and ubiquitous shall aim to address a set of design challenges. These challenges are, among others, the solution's flexibility and deployability, preferably by making use of existing infrastructure, or at least accessible technology, and the solution's security and privacy, obeying the modern cryptographic standards and requirements, to guarantee some level of privacy, and resiliency to attacks. This thesis, aiming to address these matters, delivers the following contributions:
\begin{enumerate}
\item A semi-formalization of the location-based services paradigm, including the underlying premises and the strategic interactions between rational agents, along with a review of the state of the art in the field. The review is discriminated in terms of trust levels, from fully trusted to permissionless environments, and in terms of the underlying technology, from centralized to decentralized.
\item The design and implementation of a proof-of-concept that can be deployed in a permissionless manner, using existing infrastructure, and that can be used to attest to one's presence at a given point in space and time. Specifically, the proof-of-concept is based on the use of routing protocols for multi-hop mobile ad hoc networks to set up a mesh network of witnesses that can attest to one's presence in a given geographical area.
\end{enumerate}

The structure of the work is as follows. In chapter 2, an introduction to the underlying concepts and hypotheses is provided, as well as a mention to the technology involved in the practical implementation. Chapter 3 examines and analyzes similar work discriminated in terms of trust levels. In chapter 4, a general overview of the requirements for the proposed solution is given. Chapter 5 details the architecture's design, implementation, and evaluation. Finally, chapter 6 presents the conclusion and recommendations for future work.
