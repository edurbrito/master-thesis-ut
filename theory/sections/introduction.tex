
% \TODO{What is it in simple terms (title)?}
% \TODO{Why should anyone care?}
% \TODO{What was my contribution?} 
% \TODO{What you are doing in each section (a sentence or two per section)}

Throughout history, humans have sought to accurately locate themselves, within the vastness of space and time. From mapping their surroundings and establishing their borders, to navigating the seas, with the help of stars, humans have always had an intrinsic need for fundamentally solving the localization problem, asserting their physical existence in the universe. This need has been adaptively met through various means, including the use of maps, compasses, and astronomical observations. Today, with the advent of technology, we have access to advanced methods of precise localization, like the well-known GPS-based systems. These systems are, nevertheless, founded in hinted beliefs and trust assumptions, expecting fundamental honesty, goodwill, and reliability of the all the involved subjects, when it comes to correctly reporting location information. Strategic interactions between rational agents often end up supporting this trust. One party provides a location-based service, and another party makes use of it for its individual benefit, by providing an allegedly non-tampered time-conscious piece of location claim. This interaction appears to be one of a cooperative game, that can be observed in most mapping platforms, navigation systems, mobility and ride-hailing apps, among many others. If driven by the reasoning goal of extracting correct information from the interacting system, users are logically motivated to report an accurate location. The services, having the higher goal of not losing users due to their malfunctioning or inaccuracy, are thus motivated to provide maximized quality when operating and consuming the location data. 

This paradigm is now ubiquitous, but it may be structurally unsuitable for other scenarios. Those scenarios are, therefore and inversely, the ones that fundamentally require verifiable \pol{} to assert a particular state or derive a conclusion. The trust levels required, in these scenarios, to testify to one's location claims ought as well to be crucially measurable. The concept of location-based authentication or authorization in adversarial environments that rely on information gathered in a trustless setup eventually materializes into services requiring, for instance, a digital certificate as proof that a given user is within a particular geographical area, to enable certain functionalities or assert liability. The applications of such services are numerous. These include costumer reward systems for physical stores, location-authenticated business review platforms, location-restricted web content delivery, voter's physical presence verification, and so forth. Security against geo-tampering or location spoofing in a relatively trustless environment is needed to achieve high levels of integrity. In consequence, enforcing, providing, and contributing to tamper-proof, correct and censorship resistant location information, in today's chaotically data driven world, may preemptively demand for a collective and decentralized effort. 

The basic infrastructural concept of a \pol{} system is somewhat understood, and theoretical or experimental solutions have been delivered throughout the years. These solutions have evolved parallel with their trust assumptions, beginning with a fully trusted setup and progressively shifting towards modern requirements for operational decentralization, with an inevitable distribution of resources, power, and profit. Recent attempts contemplate the need for a permissionless means of reaching consensus between a quorum of witnesses, which can attest to one's presence at a given point in space and at a given moment in time. These concepts take shape with a combination of tools: wireless technologies as short-range message-exchanging means, cryptographic protocols as confidentiality, integrity, or authentication enablers, and distributed ledgers as publicly trusted record keepers. 

The quest for a solution that could make this kind of location-based services as prevalent and ubiquitous shall aim to address a set of design challenges. These challenges are, among others, the solution's flexibility and deployability, preferably by making use of existing infrastructure, and the solution's security and privacy, obeying the modern cryptographic standards and requirements, to guarantee envisioned levels of integrity and resiliency to attacks. This thesis, aiming to address these matters, delivers the following contributions:
\begin{enumerate}
\item It provides an overview of the \pol{} systems' paradigm, including the underlying premises and the strategic interactions between rational agents, along with a review of the state of the art in the field. The review is discriminated in terms of trust levels, from fully trusted to permissionless environments, and, consequently, in terms of infrastructure, from centralized to decentralized systems.
\item It also attempts at the design and implementation of a \poc{} that can be deployed in a permissionless manner, using existing technology. The \poc{} is specifically based on the use of routing protocols for multi-hop mobile ad hoc networks. The goal is to set up a mesh network of witnesses that can achieve permissionless consensus and collectively agree to assert one's presence in a given geographical area.
\end{enumerate}

The structure of the work is as follows. In Chapter~\ref{sec:background}, an introduction to the underlying concepts, hypotheses, and applications is provided, together with a mention to the technology involved in the practical implementation. Chapter~\ref{sec:related-work} examines similar work discriminated in terms of trust levels. In Chapter~\ref{sec:protocol-fundamentals}, a general overview of the requirements for the proposed solution is given. Chapter~\ref{sec:proof-of-concept} details the architecture's design, implementation, and evaluation. Finally, Chapter~\ref{sec:conclusion} presents the conclusion and recommendations for future work.
