This section presents a nuanced description of the current state of the \pol{} problem, spanning the spectrum of its trust levels, from fully trusted to permissionless environments. Furthermore, it encompasses an assessment of the typical infrastructural scenarios, detailing the progressive shift from centralized to decentralized systems. The organization of the section is as follows. Section~\ref{sec:related-work-trusted-centralized} outlines the starting point in a trusted and centralized setting. Section~\ref{sec:related-work-distributed-decentralized} details the progressive shift towards distributed and decentralized protocols. Section~\ref{sec:related-work-fully-trustless} presents the most recent developments in the \pol{} problem, which ultimately target permissionless and fully trustless environments.


% From Group Time-based One-time Passwords and its Application to Efficient Privacy-Preserving Proof of Location:

% Waters and Felten [49] proposed the first proof of location scheme in a centralized setting

% Graham and Gray [18] proposed a scheme called SLVPGP that removes the need for the location manager

% In the decentralized settings, Zhu and Cao [52] proposed AP-PLAUS, 2011

% Alibi, 2012, Link 2012, Props 2014, STAMP 2016, SPARSE 2018

% King [31] proposes FOAM that relies on decentralized and
% trusted zone anchor beacons with synchronized clocks.

% Wu et al. [50] proposed a blockchain-based zero-knowledge proof of location scheme. The
% user obtains a location certificate from a number of location beacons, as in FOAM, and generates a zero-knowledge proof of its with the help of the beacons

% Dupin et al. [11] proposed another privacy-preserving scheme using group signatures for identity privacy, combining with secure multi-party computation for location privacy. It achieves strong privacy guarantees, but suffers high performance overhead due to the expensive cryptographic primitives.

% The existing decentralized proof of location schemes use digital signatures, which may render them impractical for running frequently on low-power devices. Furthermore, they lack formal security analysis. Our PoL based on GTOTP is efficient because it does not use digital signatures, and it is provably secure. ---- No time synchrnization, or it may need to rely on an external trusted time source: Group Time-based One-time Passwords

% --------------------------------

% Privacy-preserving Proof-of-Location With Security Against Geo-tampering:

% Related work section

% --------------------------------

% Robust Decentralized Proof of Location for Blockchain Energy Applications Using Game Theory and Random Selection

% Background work section


% --------------------------------

% Blockchain-based Proof of Location

% Related Work section


% --------------------------------

% A Robust Spatio-Temporal Verification Protocol for Blockchain

% Related Work


% -----------------------------------

% Location-Proof System based on Secure Multi-Party Computations

% Introduction

% -----------------------------------

% Blockchain for secure location verification

% Related work

% -----------------------------------

