The problem of attesting to one's location is a fundamental act of metaphysical reasoning that happens everywhere, at every moment. Unconsciously and unwittingly, we do claim to be somewhere at an indiscriminated point in time, and we do expect others to believe in us. This act is, however, grounded on informal and implicit levels of trust that are not explicitly asserted, as liability is often not categorically assigned. When it does happen, trust is usually delegated to a third party or distributed between multiple parties, that may be able to testify to one's presence, synchronously, at the very same location. The act of witnessing is, therefore, a regular yet fundamental part of our interactions with physical reality. When we do claim our presence at an event, report our alibi to authorities, or assert our location to a service provider, a protocol for location attestation is implicitly followed. Some may require a physical interaction of any kind, while others may find digitalized and infrastructural means to gather the required location proof \cite{luo2010veriplace}.

A digital \pol can then be defined as an electronic certificate that assuredly attests one's relative position in both space and time \cite{amoretti2018blockchain}. The relativity of the attestation is, however, not a trivial matter. It is, in fact, a complex and multi-faceted process that requires the simultaneous existence of various untrusted or semi-trusted parties, especially in an environment with no individual honesty guarantees. According to \cite{nasrulin2018robust}, a \pol protocol may be considered secure if complete, spatiotemporally sound, non-transferable and tamper-evident. The system that materially backs the implementation of such a protocol is, therefore, expected to provide fault-tolerance, reliability, and availability guarantees. More advanced protocols may also explore the possibility of providing privacy and anonymity assurances \cite{li2020privacy}, as well as the possibility of being used in a fully trustless environment \cite{amoretti2018blockchain}. Following is the conceptualization of the common entities of a \pol protocol.

\subsubsection{Parties Involved}

The general act of witnessing alludes to the simultaneous spatiotemporal existence of a set of entities with distinct roles. The majority of the protocols convey a clear contrast between these roles, highlighting the relative dynamism that differentiates those entities. 

In comparable terms, highly dynamic entities do not maintain a fixed geographical location for long periods of time. They are often observed in movement, thereby repeatedly starting and finishing communication procedures with neighbouring entities. On the other hand, static entities are expected not to engage in frequent position changes, expressing continuous and fairly invariable communication availability around a fixed point in space as time passes \cite{nasrulin2018robust}. The act is, however, only completed with another type of entity from whom neither the relative staticity nor the relative dynamism frankly matters. These protocol parties are often external and asynchronous to the witnessing process, but they do effectively take a non-negligible part in incentivizing and giving significance to the witnessing act. 

Concisely and in concrete terms, these location-proof arrangements expect the existence of a \emph{prover} that engages in any communication protocol with nearby participants, the \emph{witnesses}, with the goal of gathering a verifiable \pol claim, to be later presented to a \emph{verifier}, therefore convincing it of one's existence within a geographical area, at a given moment \cite{dupin2018location}.

\paragraph{Prover.} A prover is a dynamic entity, both in movement and availability terms, that is expected to be able to communicate with the witnesses, to gather a proof of its location, and to be later able to provide a location claim to the verifier. Communication with nearby witnesses is thought to happen wirelessly, using any short-range message transmission means. Provers are also expected to be associated with a verifiable but desirably private identity, often as a pseudonym.

\paragraph{Witness.} A witness is an entity that is expected to be able to communicate with the prover via the same short-range communication channel and to provide it with a verifiable piece of location attestation. The witnesses are envisioned to seldomly change their absolute location and maintain a relatively stable neighbouring list of nearby witnesses. These references aim at attaining the figurative creation of coverage zones as strongly connected graphs that form the boundaries of the atomic units of a polygonal mesh. Witnesses are as well expected to be identified, usually by a pseudonym.

\paragraph{Verifier.} A verifier is an external entity that is able to receive a location claim from a prover and verify its validity. Although possible and predicted for trusted setups, in a trustless environment and with the general assurances of a permissionless protocol, verifiers shall not have the need to communicate directly with the witnesses. Verifiers' identity is also of no measurable importance for the protocol, as the interaction between the prover and the verifier is usually asynchronous and external to the witnessing process.

\TODO{Should I introduce formal definitions, like \cite{nasrulin2018robust}: 
\\ Definition 1 (\pol). A \pol is a verifiable digital certificate that attests the presence of a prover $\sigma$ at location $l$ and time $t$ and is signed by an authorised witness $\omega$.
\\ And for Completeness, Spatio-Temporally soundness, non-transferability, tamper-evidence...?}

\subsubsection{Common Threat Models}

As with any technology that involves the collection and processing of sensitive and tamper-prone location data, \pol systems must be designed and implemented with a keen awareness of the threat landscape. The threat models of these systems are very often intricately multisided, encompassing a diverse range of actors, motives, and attack vectors. In this context, it is crucial to understand not only the technical mechanisms of \pol systems, but also the broader factors that shape their security and privacy risks. 

Some common scenarios that may affect the security of \pol systems are, for instance, malicious provers that may attempt to forge location claims, or witnesses that may attempt to collude with other entities to falsify the information. Adversary efforts may also be observed in the form of baleful provers, or witnesses, that may try to respectively impersonate other peers. Sybil attacks are also on the horizon of possible threats, often employed to disrupt the operation of the system by flooding it with fake participants \cite{nasrulin2018robust}. Other works have also considered semi-honest adversaries that, despite following the protocol rules, may try to learn additional information from the messages exchanged \cite{dupin2018location}.