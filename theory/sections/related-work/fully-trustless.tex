Inspired by the solution proposed by Zhu and Cao \cite{zhu2011applaus}, Amoretti et al. \cite{amoretti2018blockchain} dive into the definition of a novel decentralized and infrastructure-independent approach that allies together short-range communication technology and Blockchain-based storage and information verification. The authors propose the establishment of a distributed overlay network of linked nodes that, at the same time, wirelessly provide or request location proofs from nearby nodes, and verify or store propagated proofs, via any typical lower-level Blockchain protocolar agreement, achieving, thus, permissionless consensus. Their solution is claimed to be one of the very first at protecting against the main location-based-systems' attacks, with the help of a fully decentralized and Blockchain inspired peer-to-peer scheme, which assures both integrity and user privacy. Real-world performance evaluation and the possibility for integrating higher-level incentive mechanisms were set as future work prospects. Both Amoretti et al. \cite{amoretti2018blockchain} and Nasrulin et al. \cite{nasrulin2018robust} contemporaneous works illustrate practical constructs that take advantage of the tamper and censorship resistent nature of Blockchain technology. The latter tries, as well, to formalize the main security and spacio-temporal requirements that such a decentralized \pol{} protocol shall present, ending up implementing a \poc{}, based on a permissioned Blockchain framework, to specifically solve supply chain tracking challenges.

Further efforts that build upon the above-mentioned solutions are the ones proposed by Wu et al. \cite{wu2020blockchain} and Nosouhi et al. \cite{nosouhi2020blockchain}. The first follows the path of Amoretti et al. \cite{amoretti2018blockchain} and tries to enable, on top of it, user-defined hierarchical privacy protection, with the help of Zero-Knowledge proofs. The proposed protocol finds a bridge between the typical \pol{} set of entities and the usual Zero-Knowledge proof participants. The suggested \emph{zk-PoL} protocol aims so at allowing the \emph{prover} to convince the \emph{verifier} that one was at a specific location, at a certain point in time, but with a granular privacy preserving disclosure of the location proof details. The obvious motivation of the mechanism is to solve spam, traceability, and privacy concerns of publicly storing raw location information, especially within decentralized and public ledgers. Therefore, the scheme is, to a great degree, centred in the privacy assurances and not in the infrastructural aspects of the potential decentralization that it is built upon. Nevertheless, it sets a promising starting point for the introduction of privacy preserving technology in the realms of trustless \pol{} protocols. Optimizations and faster proof mechanisms are kept in the outlook and waiting to be explored. Nosouhi et al. \cite{nosouhi2020blockchain} stress out a different proximity checking mechanism, to protect against the still unsolved \emph{prover} and \emph{witnesses} collusions, while committing, as well, to privacy preserving location proof generation and storage, using public and decentralized Blockchain technology. Their work has also an original integration of an incentive mechanism that rewards collaborative participants, in order to more strongly prevent the main known attacks. This sets an unprecedented track for the incorporation of these \pol{} protocols into the digital and decentralized economy that already runs, via Smart Contracts, on Blockchain networks like Ethereum \cite{nosouhi2020blockchain, buterin2014next}. 

Minding all the above, Pournaras \cite{pournaras2020proof} proposes the complementing concept of Proof-of-Witness-Presence as a key element in an augmented democracy approach to smart city development. This concept involves validating the accuracy of data collected through participatory crowd-sensing, by requiring physical presence at locations of interest. The author argues that this approach can foster greater citizen engagement and participation in public spaces, and can be incentivized through Blockchain consensus and a crypto-economic design. Acknowledging the limitations of current localization methods such as GPS, it is suggested the need for more advanced and secure location certificates, based on complex social proofs. The Proof-of-Witness-Presence model envisioned by Pournaras may rely on token curated registries and a supplemental fully trustless \pol{} protocol that, for instance, FOAM\footnote{\url{https://foam.space/}} tries to deliver. The next paragraph will be fully dedicated to this last piece of work.

