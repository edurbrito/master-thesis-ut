Inspired by the solution proposed by Zhu and Cao \cite{zhu2011applaus}, Amoretti et al. \cite{amoretti2018blockchain} dive into the definition of a novel decentralized and infrastructure-independent approach that allies together short-range communication technology and Blockchain-based storage and information verification. The authors propose the establishment of a distributed overlay network of linked nodes that, at the same time, wirelessly provide or request location proofs from nearby nodes, and verify or store propagated proofs, via any typical lower-level Blockchain protocolar agreement, achieving, thus, permissionless consensus. Their solution is claimed to be one of the very first at protecting against the main location-based-systems attacks, with the help of a fully decentralized and blockchain inspired peer-to-peer scheme, which assures both integrity and user privacy. Real-world performance evaluation and the possibility for integrating higher-level incentive mechanisms were set as future work prospects. Both Amoretti et al. \cite{amoretti2018blockchain} and Nasrulin et al. \cite{nasrulin2018robust} contemporaneous works illustrate practical constructs that take advantage of the tamper and censorship resistent nature of Blockchain technology. The latter tries, as well, to formalize the main spacio-temporal and security requirements that such a decentralized \pol{} protocol shall present, ending up implementing a \poc{} based on a permissioned Blockchain framework, to specifically solve supply chain tracking challenges.

Further efforts that build upon the above-mentioned solutions, are the ones proposed by 