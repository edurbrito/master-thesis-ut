VeriPlace \cite{luo2010veriplace} had already profiled and templated an inherently distributed architecture, with built-in privacy awareness, taking a first infrastructural step towards defending against proxy attacks, without the need for trusted hardware. The whole setup is especially tangled and consequently resourceful for the levels of trust it assumes, but it definitely settled the ground for the next generation of \pol{} schemes. 

The following evolutionary stage of these protocols aims at flexing and distributing trust, resources, power and responsibility, with the hope of achieving more resilient, fault-tolerant and scalable systems. APPLAUS, by Zhu and Cao \cite{zhu2011applaus}, delivers one of the first distributed protocols that combines the location proof and location privacy problems. It uses Bluetooth enabled mobile devices that communicate with nearby participants, during the proof generation process. The protocol asserts certain bond levels between the \emph{prover}, \emph{verifier}, and \emph{witnesses}, all of them known to a trusted Certificate Authority (CA), disregarding, on the other hand, the need for a fully trusted location proof server to store the historic location records. The claim is that, by statistically changing the pseudonyms for each device and by following a user-centric privacy model, the protocol can effectively generate privacy preserving location proofs and store them in a trustless manner. STAMP \cite{wang2016stamp} and PROPS \cite{gambs2014props} are two contemporaneous works that take the same witnessing approach as APPLAUS, but follow the path of convincing the \emph{verifier} by presenting several shares of a composite location proof, based on group signatures. Both of them try to more profoundly tackle the \emph{prover}'s and \emph{witnesses}' privacy concerns, but may admittedly fail at preventing collision scenarios between them. Gambs et al. argue that the reliance on a trusted third party may be an unavoidable requirement, even if against the authors' principles of location sovereignty, if one wants to entirely prevent unbounded collusion attacks \cite{gambs2014props}. SPARSE, by Nosouhi et al. \cite{nosouhi2018sparse}, avoids the typical distance-bounding and \emph{witness} picking process by the \emph{prover}, with the goal of protecting against those collusion attempts, at best, in relatively crowded situations.

