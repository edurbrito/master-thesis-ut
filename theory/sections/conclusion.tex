The \pol{} problem was the main target of this thesis. Through theoretical and practical contributions, we attempted at dissecting its roots. The work started with the identification of the underlying concepts, hypotheses, and real-world applications, and was followed by a review of the state of the art, encompassing a wide range of trust levels and infrastructural scenarios. Inspired by previous work, this thesis delivered a novel approach to the problem. The proposed solution involved the specification of a decentralized \pol{} protocol and the implementation of a \poc{}. The final work mixes routing protocols for mobile ad hoc mesh networks and permissionless consensus mechanisms, in order to finally achieve collective agreement on one's presence in a specific location, at a certain point in time. We demonstrated the protocol using a network architecture that is both distributed and modular, where each stage has its own distinct set of responsibilities. The chosen technologies showed optimistic results that make the work suitable and adaptable to resource-constrained environments. We believe the proposed protocol is a promising step towards achieving Absolute \pol{}, in a decentralized and trustless manner.

The groundwork for the implementation of a novel \pol{} protocol has been established, but further investigation is still needed. For instance, we are looking forward to identifying the most effective identity management systems and crypto-economic incentives to motivate nodes to collaborate and establish and maintain coverage zones, with the research on the processes of zone discovery and zone affinity management revealing to be essential for the overall success of the protocol. Another aspect that deserves further investigation is the shift from deterministic-finality Byzantine fault-tolerant algorithms to probabilistic finality consensus mechanisms. Formal verifications, measurements, and comparisons between the two approaches are necessary to determine the best candidates for decentralized \pol{}. Furthermore, we have demonstrated the simplest case of applying a consensus mechanism to achieve time synchronization. Extending the approach to reach more accurate location attestations, making full use of the system's Turing Completeness for more complex logic, and ensuring the extensibility of the approach are all areas for future work. A thorough analysis of the robustness, security, privacy, and correctness of the baseline solution is also to be presented. We point our interests as well towards the integration of privacy preserving mechanisms, such as zero-knowledge proofs. Lastly, the live deployment of the solution and the evaluation of the performance in real-world scenarios, contributing to the production of more secure and tamper-proof location data, is the ultimate step we are aiming for.

\newpage

\subsection*{Acknowledgments}

I would like to express gratitude towards my supervisors, for all the guidance, feedback, and support provided throughout the development of this thesis. All the other people that have contributed to the reasoning and validation of the work are also acknowledged.

Words of appreciation should go, as well, to the multiple tools that aided the writing and development of this thesis. ChatGPT, Grammarly, Code Spell Checker, LTeX, and GitHub Copilot provided invaluable support throughout the working process, without discrediting all the other tools, extensions, and development environments that, directly or indirectly, with or without AI features, contributed to the efficiency of the work. Appendix~\ref{appendix:writing-workflow} provides a detailed description of the writing workflow and the tools used to produce this thesis.

Finally, I would like to thank the University of Tartu for providing multiple opportunities and access to resources and funding, supporting the development of this work. In particular, I would like to mention the IT Academy, the XRP Ledger Trust, and Cybernetica AS, for the financial support provided.